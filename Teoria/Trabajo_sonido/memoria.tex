\documentclass[11pt,a4paper]{article}
\usepackage[utf8]{inputenc}
\usepackage[spanish]{babel}	%Idioma
\usepackage{amsmath}
\usepackage{amsfonts}
\usepackage{amssymb}
\usepackage{graphicx} 	%Añadir imágenes
\usepackage{geometry}	%Ajustar márgenes
\usepackage[export]{adjustbox}[2011/08/13]
\usepackage{float}
\restylefloat{table}
\usepackage{diagbox}
\usepackage{makecell}
\usepackage[hidelinks]{hyperref}
\usepackage{titling}
%\graphicspath{{/home/nazaret/Escritorio/LaTEX}}
%\usepackage{minted}
\usepackage{multirow}
\usepackage{caption}
\usepackage{multicol}
\usepackage{array}
\selectlanguage{spanish}
\usepackage{xcolor}

%Opciones de encabezado y pie de página:
\usepackage{fancyhdr}
\pagestyle{fancy}
\lhead{Nazaret Román Guerrero}
\rhead{Sistemas Multimedia}
\lfoot{Grado en Ingeniería Informática}
\cfoot{}
\rfoot{\thepage}
\renewcommand{\headrulewidth}{0.4pt}
\renewcommand{\footrulewidth}{0.4pt}

%Opciones de fuente:
\usepackage[utf8]{inputenc}
\usepackage[default]{sourcesanspro}
\usepackage{sourcecodepro}
\usepackage[T1]{fontenc}

\setlength{\parindent}{15pt}
\setlength{\headheight}{15pt}
\setlength{\voffset}{10mm}

% Custom colors
\usepackage{color}
\definecolor{deepblue}{rgb}{0,0,0.5}
\definecolor{deepred}{rgb}{0.6,0,0}
\definecolor{deepgreen}{rgb}{0,0.5,0}

\usepackage{listings}

\begin{document}
\begin{titlepage}

\begin{minipage}{\textwidth}

\centering
\includegraphics[width=0.5\textwidth]{logo.png}\\

\textsc{\Large Sistemas Multimedia\\[0.2cm]}
\textsc{GRADO EN INGENIERÍA INFORMÁTICA}\\[1cm]

{\Huge\bfseries Sonido\\}
\noindent\rule[-1ex]{\textwidth}{3pt}\\[3.5ex]
{\large\bfseries Trabajo voluntario}
\end{minipage}

\vspace{1.5cm}
\begin{minipage}{\textwidth}
\centering

\textbf{Autora}\\ {Nazaret Román Guerrero}\\[2.5ex]
\includegraphics[width=0.3\textwidth]{etsiit.jpeg}\\[0.1cm]
\vspace{1cm}
\textsc{Escuela Técnica Superior de Ingenierías Informática y de Telecomunicación}\\
\vspace{1cm}
\textsc{Curso 2018-2019}
\end{minipage}
\end{titlepage}

\pagenumbering{gobble}
\pagenumbering{arabic}
\tableofcontents
\thispagestyle{empty}

\newpage

\section{Parámetros de digitalización}

Para comparar la calidad de audio y relacionarlo con el tamaño, en esta sección cambiaremos la frecuencia de muestreo y la resolución de distintas pistas de audio:

\begin{enumerate}
	\item Grabación de voz
	\item \textit{Power Ballad}
	\item Rock progresivo
\end{enumerate}

En cada pista se creará una tabla en cuyas celdas habrá un valor numérico con su unidad, asociado al tamaño que ocupa la pista con esa resolución y ese muestreo, y un icono con forma de cara, con distintas expresiones y de distinto color, asociado a la calidad de audio, de forma que ver la tabla se haga un poco más visual que tan solo leer texto y cifras.\\

La leyenda de iconos es la siguiente:

\begin{itemize}
	\item Muy buena = \includegraphics[width=0.05\textwidth]{mb.png}
	\item Buena = \includegraphics[width=0.05\textwidth]{b.png}
	\item Regular = \includegraphics[width=0.05\textwidth]{r.png}
	\item Mala = \includegraphics[width=0.05\textwidth]{m.png}
	\item Muy mala = \includegraphics[width=0.05\textwidth]{mm.png}
\end{itemize}

\subsection{Grabación de voz}

En la primera prueba, utilizaremos una grabación de voz en la que se lee un fragmento del libro \textit{Juego de Tronos}.

\begin{itemize}
	\item Formato original: \textsc{.aac}
	\item Formato: \textsc{.wav}
	\item Tamaño original: 7.6 MB
\end{itemize}

\begin{table}[H]
\centering
\begin{tabular}{|c|c|c|c|}
\hline
\diagbox[width=15em]{Frecuencia\\de muestreo}{\\Resolución} & 32 bits & 16 bits & 8 bits  \\ \hline

8000 Hz (8 kHz) & \makecell{2.8 MB\\ \includegraphics[width=0.03\textwidth]{m.png}} & \makecell{1.4 MB\\ \includegraphics[width=0.03\textwidth]{m.png}} & \makecell{692.4 kB\\ \includegraphics[width=0.03\textwidth]{mm.png}} \\ \hline

11025 Hz (11.025 kHz) & \makecell{3.8 MB\\ \includegraphics[width=0.03\textwidth]{r.png}} & \makecell{1.9 MB\\ \includegraphics[width=0.03\textwidth]{r.png}} & \makecell{954.2 kB\\ \includegraphics[width=0.03\textwidth]{m.png}} \\ \hline

22050 Hz (22.05 kHz) & \makecell{7.6 MB\\ \includegraphics[width=0.03\textwidth]{mb.png}} & \makecell{3.8 MB\\ \includegraphics[width=0.03\textwidth]{mb.png}} & \makecell{1.9 MB\\ \includegraphics[width=0.03\textwidth]{m.png}} \\ \hline

44100 Hz (44.1 kHz) & \makecell{15.3 MB\\ \includegraphics[width=0.03\textwidth]{mb.png}} & \makecell{7.6 MB\\ \includegraphics[width=0.03\textwidth]{mb.png}} & \makecell{3.8 MB\\ \includegraphics[width=0.03\textwidth]{r.png}} \\ \hline

48000 Hz (48 kHz) & \makecell{16.6 MB\\ \includegraphics[width=0.03\textwidth]{mb.png}} & \makecell{8.3 MB\\ \includegraphics[width=0.03\textwidth]{mb.png}} & \makecell{4.2 MB\\ \includegraphics[width=0.03\textwidth]{r.png}} \\ \hline

72000 Hz (72 kHz) & \makecell{24.9 MB\\ \includegraphics[width=0.03\textwidth]{mb.png}} & \makecell{12.5 MB\\ \includegraphics[width=0.03\textwidth]{mb.png}} & \makecell{6.2 MB\\\includegraphics[width=0.03\textwidth]{r.png}} \\ \hline
\end{tabular}
\caption{Tabla comparativa}
\label{my-label}
\end{table}

Hay tres parejas que mantienen la calidad del audio original y que ocupan menos tamaño. Son las siguientes:

\begin{itemize}
	\item 22.05 kHz y 16 bits de resolución. Ocupa 3.8 MB.
	\item 22.05 kHz y 32 bits de resolución. Ocupa 7.6 MB, por lo que es mejor la primera opción.
	\item 44.1 kHz y 16 bits de resolución. Al igual que la anterior, ocupa 7.6 MB, por lo que la primera opción sigue siendo la mejor.
\end{itemize}

\subsection{Nothing else matters}

Para esta segunda prueba, utilizaremos una canción extraída de un CD (\textsc{.cda}), que ha sido convertida a \textsc{.wav} y sobre la que realizaremos los cambios.

\begin{itemize}
	\item Canción: Nothing else matters
	\item Artista: Metallica
	\item Género: \textit{Power Ballad}
	\item Formato  original: \textsc{.cda}
	\item Formato: \textsc{.wav}
	\item Tamaño original: 68.1 MB
\end{itemize}

\begin{table}[H]
\centering
\begin{tabular}{|c|c|c|c|}
\hline
\diagbox[width=15em]{Frecuencia\\de muestreo}{\\Resolución} & 32 bits & 16 bits & 8 bits  \\ \hline

8000 Hz (8 kHz) & \makecell{27.4 MB\\ \includegraphics[width=0.03\textwidth]{mm.png}} & \makecell{12.3 MB\\ \includegraphics[width=0.03\textwidth]{mm.png}} & \makecell{6.2 MB\\ \includegraphics[width=0.03\textwidth]{mm.png}} \\ \hline

11025 Hz (11.025 kHz) & \makecell{34 MB\\ \includegraphics[width=0.03\textwidth]{r.png}} & \makecell{17 MB\\ \includegraphics[width=0.03\textwidth]{r.png}} & \makecell{8.5 MB\\ \includegraphics[width=0.03\textwidth]{m.png}} \\ \hline

22050 Hz (22.05 kHz) & \makecell{68.1 MB\\ \includegraphics[width=0.03\textwidth]{mb.png}} & \makecell{34 MB\\ \includegraphics[width=0.03\textwidth]{b.png}} & \makecell{17 MB\\ \includegraphics[width=0.03\textwidth]{r.png}} \\ \hline

44100 Hz (44.1 kHz) & \makecell{136.1 MB\\ \includegraphics[width=0.03\textwidth]{mb.png}} & \makecell{68.1 MB\\ \includegraphics[width=0.03\textwidth]{mb.png}} & \makecell{34 MB\\ \includegraphics[width=0.03\textwidth]{r.png}} \\ \hline

48000 Hz (48 kHz) & \makecell{148.1 MB\\ \includegraphics[width=0.03\textwidth]{mb.png}} & \makecell{74.1 MB\\ \includegraphics[width=0.03\textwidth]{mb.png}} & \makecell{37 MB\\ \includegraphics[width=0.03\textwidth]{r.png}} \\ \hline

72000 Hz (72 kHz) & \makecell{222.2 MB\\ \includegraphics[width=0.03\textwidth]{mb.png}} & \makecell{111.1 MB\\ \includegraphics[width=0.03\textwidth]{mb.png}} & \makecell{55.6 MB\\\includegraphics[width=0.03\textwidth]{r.png}} \\ \hline
\end{tabular}
\caption{Tabla comparativa}
\label{my-label}
\end{table}

Como podemos comprobar, en este ejemplo, las parejas que mantienen la calidad del audio original y que ocupan menos tamaño (ambas ocupan 68.1 MB) son:

\begin{itemize}
	\item 22.05 kHz y 32 bits de resolución.
	\item 44.1 kHz y 16 bits de resolución.
\end{itemize}

\subsection{The logical song}

Para la última prueba, utilizaremos otra canción de un CD convertida a \textsc{.wav}.

\begin{itemize}
	\item Canción: The logical song
	\item Artista: Supertramp
	\item Género: Rock progresivo
	\item Formato  original: \textsc{.cda}
	\item Formato: \textsc{.wav}
	\item Tamaño original: 43.9 MB
\end{itemize}

\begin{table}[H]
\centering
\begin{tabular}{|c|c|c|c|}
\hline
\diagbox[width=15em]{Frecuencia\\de muestreo}{\\Resolución} & 32 bits & 16 bits & 8 bits  \\ \hline

8000 Hz (8 kHz) & \makecell{15.9 MB\\ \includegraphics[width=0.03\textwidth]{mm.png}} & \makecell{8 MB\\ \includegraphics[width=0.03\textwidth]{mm.png}} & \makecell{4 MB\\ \includegraphics[width=0.03\textwidth]{mm.png}} \\ \hline

11025 Hz (11.025 kHz) & \makecell{22 MB\\ \includegraphics[width=0.03\textwidth]{m.png}} & \makecell{11 MB\\ \includegraphics[width=0.03\textwidth]{m.png}} & \makecell{5.5 MB\\ \includegraphics[width=0.03\textwidth]{mm.png}} \\ \hline

22050 Hz (22.05 kHz) & \makecell{43.9 MB\\ \includegraphics[width=0.03\textwidth]{b.png}} & \makecell{22 MB\\ \includegraphics[width=0.03\textwidth]{b.png}} & \makecell{11 MB\\ \includegraphics[width=0.03\textwidth]{r.png}} \\ \hline

44100 Hz (44.1 kHz) & \makecell{87.8 MB\\ \includegraphics[width=0.03\textwidth]{mb.png}} & \makecell{43.9 MB\\ \includegraphics[width=0.03\textwidth]{mb.png}} & \makecell{22 MB\\ \includegraphics[width=0.03\textwidth]{r.png}} \\ \hline

48000 Hz (48 kHz) & \makecell{95.6 MB\\ \includegraphics[width=0.03\textwidth]{mb.png}} & \makecell{47.8 MB\\ \includegraphics[width=0.03\textwidth]{mb.png}} & \makecell{23.9 MB\\ \includegraphics[width=0.03\textwidth]{r.png}} \\ \hline

72000 Hz (72 kHz) & \makecell{143.4 MB\\ \includegraphics[width=0.03\textwidth]{mb.png}} & \makecell{71.7 MB\\ \includegraphics[width=0.03\textwidth]{mb.png}} & \makecell{35.8 MB\\\includegraphics[width=0.03\textwidth]{r.png}} \\ \hline
\end{tabular}
\caption{Tabla comparativa}
\label{my-label}
\end{table}

En este caso, la pista de audio que mejor se escucha y tiene un menor tamaño es la original.

\subsection{Conclusiones}

Como se ha observado en las tablas anteriores, la resolución de 8 bits siempre es muy mala, mala o regular en el mejor caso, ya que en todas las pistas de audio se escucha un zumbido muy molesto que hace que con esta resolución pierda mucha calidad y no sea agradable escuchar la pista de audio. Aunque el resto de la melodía y las voces se escuchen bien, ese ruido resulta muy desagradable y, aunque se mantenga la calidad del resto del audio, se hace imposible poner una buena calificación.\\

Además, en cada pista de las tres diferentes, las tablas varían, especialmente la grabación de voz con respecto a las otras dos, tal vez debido a la calidad original del propio audio, lo que nos lleva a que la grabación, que originalmente estaba comprimida con el códec \textsc{aac}, tenga mucha diferencia con respecto a los audios que no tenían ningún tipo de compresión.

\newpage

\section{Comparativa de \textit{códecs}}

En esta sección se probarán distintos \textit{códecs} sobre pistas de audio para comprobar las diferencias de calidad.

\subsection{The Unforgiven}

\begin{itemize}
	\item Canción: The Unforgiven
	\item Artista: Metallica
	\item Género: \textit{Power Ballad}
	\item Frecuencia de muestreo: 44.1 kHz
	\item Resolución: 16 bits
	\item Duración: 6:18
\end{itemize}

\begin{table}[H]
\centering
\begin{tabular}{|c|c|c|c|c|c|}
\hline
\diagbox[width=15em]{Resolución}{\\\textit{Códec}} & \textsc{ mp3 } & \textsc{ aac } & \textsc{ wma } & \textsc{vorbis} & \textsc{ opus } \\ \hline

32 bits & \makecell{2º\\ \includegraphics[width=0.03\textwidth]{mb.png}} & \makecell{1º\\ \includegraphics[width=0.03\textwidth]{mb.png}} & \makecell{2º\\ \includegraphics[width=0.03\textwidth]{mb.png}} & \makecell{1º\\ \includegraphics[width=0.03\textwidth]{mb.png}} & \makecell{1º\\ \includegraphics[width=0.03\textwidth]{mb.png}} \\ \hline
 
16 bits & \makecell{2º\\ \includegraphics[width=0.03\textwidth]{mb.png}} & \makecell{1º\\ \includegraphics[width=0.03\textwidth]{mb.png}} & \makecell{2º\\ \includegraphics[width=0.03\textwidth]{mb.png}} & \makecell{1º\\ \includegraphics[width=0.03\textwidth]{mb.png}} & \makecell{1º\\ \includegraphics[width=0.03\textwidth]{mb.png}} \\ \hline
 
 8 bits & \makecell{3º\\ \includegraphics[width=0.03\textwidth]{m.png}} & \makecell{2º\\ \includegraphics[width=0.03\textwidth]{mb.png}} & \makecell{4º\\ \includegraphics[width=0.03\textwidth]{m.png}} & \makecell{1º\\ \includegraphics[width=0.03\textwidth]{mb.png}} & \makecell{1º\\ \includegraphics[width=0.03\textwidth]{mb.png}} \\ \hline
\end{tabular}
\caption{Tabla comparativa}
\label{my-label}
\end{table}

\subsubsection{Explicaciones a las valoraciones}
\begin{enumerate}
	\item \textsc{mp3} y \textsc{wma}, 32 bits y 16 bits: se escuchan iguales a la original, han sido colocadas como las 2ª opciones solo por cómo se comportan con una resolución de 8 bits. Pero se escuchan perfectas.
	\item \textsc{aac, vorbis} y \textsc{opus}, 32 bits y 16 bits: iguales a la original.
	\item \textsc{vorbis} y \textsc{opus}, 8 bits: no se pierde nada de calidad de audio, se escuchan exactamente igual a la original.
	\item \textsc{aac}, 8bits: casi igual a la original. Se escucha algo más grave, como si hubiesen tapado parcialmente el micrófono.
	\item \textsc{mp3}, 8 bits: no se escucha bien, aunque sí algo mejor que \textsc{wma}.
	\item \textsc{wma}, 8bits: el que peor se escucha, con diferencia. Hay ruido de fondo.
\end{enumerate}

\subsection{Nothing else matters}

\begin{itemize}
	\item Canción: Nothing else matters
	\item Artista: Metallica
	\item Género: \textit{Power Ballad}
	\item Frecuencia de muestreo: 44.1 kHz
	\item Resolución: 16 bits
	\item Duración: 6:26
\end{itemize}

\begin{table}[H]
\centering
\begin{tabular}{|c|c|c|c|c|c|}
\hline
\diagbox[width=15em]{Resolución}{\\\textit{Códec}} & \textsc{ mp3 } & \textsc{ aac } & \textsc{ wma } & \textsc{vorbis} & \textsc{ opus } \\ \hline

32 bits & \makecell{2º\\ \includegraphics[width=0.03\textwidth]{mb.png}} & \makecell{1º\\ \includegraphics[width=0.03\textwidth]{mb.png}} & \makecell{2º\\ \includegraphics[width=0.03\textwidth]{mb.png}} & \makecell{1º\\ \includegraphics[width=0.03\textwidth]{mb.png}} & \makecell{1º\\ \includegraphics[width=0.03\textwidth]{mb.png}} \\ \hline
 
16 bits & \makecell{2º\\ \includegraphics[width=0.03\textwidth]{mb.png}} & \makecell{1º\\ \includegraphics[width=0.03\textwidth]{mb.png}} & \makecell{2º\\ \includegraphics[width=0.03\textwidth]{mb.png}} & \makecell{1º\\ \includegraphics[width=0.03\textwidth]{mb.png}} & \makecell{1º\\ \includegraphics[width=0.03\textwidth]{mb.png}} \\ \hline
 
 8 bits & \makecell{3º\\ \includegraphics[width=0.03\textwidth]{r.png}} & \makecell{2º\\ \includegraphics[width=0.03\textwidth]{mb.png}} & \makecell{4º\\ \includegraphics[width=0.03\textwidth]{m.png}} & \makecell{1º\\ \includegraphics[width=0.03\textwidth]{mb.png}} & \makecell{1º\\ \includegraphics[width=0.03\textwidth]{mb.png}} \\ \hline
\end{tabular}
\caption{Tabla comparativa}
\label{my-label}
\end{table}

\subsubsection{Explicaciones a las valoraciones}
\begin{enumerate}
	\item \textsc{mp3} y \textsc{wma}, 32 bits y 16 bits: se escuchan iguales a la original, y, al igual que la canción anterior, han sido colocadas como las 2ª por cómo se escuchan con una resolución de 8 bits.
	\item \textsc{aac, vorbis} y \textsc{opus}, 32 bits y 16 bits: iguales a la original.
	\item \textsc{vorbis} y \textsc{opus}, 8 bits: no se pierde nada de calidad de audio, se escuchan exactamente igual a la original.
	\item \textsc{aac}, 8bits: casi igual a la original. Se escucha algo más grave, como si hubiesen tapado parcialmente el micrófono.
	\item \textsc{mp3}, 8 bits: se escucha bastante mejor que la canción del caso anterior para esta misma resolución, pero sigue perdiendo calidad con respecto a la original.
	\item \textsc{wma}, 8bits: se escucha un zumbido, el sonido no esta para nada claro.
\end{enumerate}

\subsection{The logical song}

\begin{itemize}
	\item Canción: The logical song
	\item Artista: Supertramp
	\item Género: Rock progresivo
	\item Frecuencia de muestreo: 44.1 kHz
	\item Resolución: 16 bits
	\item Duración: 4:10
\end{itemize}

\begin{table}[H]
\centering
\begin{tabular}{|c|c|c|c|c|c|}
\hline
\diagbox[width=15em]{Resolución}{\\\textit{Códec}} & \textsc{ mp3 } & \textsc{ aac } & \textsc{ wma } & \textsc{vorbis} & \textsc{ opus } \\ \hline

32 bits & \makecell{2º\\ \includegraphics[width=0.03\textwidth]{b.png}} & \makecell{2º\\ \includegraphics[width=0.03\textwidth]{b.png}} & \makecell{2º\\ \includegraphics[width=0.03\textwidth]{b.png}} & \makecell{1º\\ \includegraphics[width=0.03\textwidth]{mb.png}} & \makecell{1º\\ \includegraphics[width=0.03\textwidth]{mb.png}} \\ \hline
 
16 bits & \makecell{2º\\ \includegraphics[width=0.03\textwidth]{b.png}} & \makecell{2º\\ \includegraphics[width=0.03\textwidth]{b.png}} & \makecell{2º\\ \includegraphics[width=0.03\textwidth]{b.png}} & \makecell{1º\\ \includegraphics[width=0.03\textwidth]{mb.png}} & \makecell{1º\\ \includegraphics[width=0.03\textwidth]{mb.png}} \\ \hline
 
 8 bits & \makecell{3º\\ \includegraphics[width=0.03\textwidth]{m.png}} & \makecell{2º\\ \includegraphics[width=0.03\textwidth]{b.png}} & \makecell{4º\\ \includegraphics[width=0.03\textwidth]{mm.png}} & \makecell{1º\\ \includegraphics[width=0.03\textwidth]{mb.png}} & \makecell{1º\\ \includegraphics[width=0.03\textwidth]{mb.png}} \\ \hline
\end{tabular}
\caption{Tabla comparativa}
\label{my-label}
\end{table}

\subsubsection{Explicaciones a las valoraciones}
\begin{enumerate}
	\item \textsc{mp3, aac} y \textsc{wma}, 32 bits y 16 bits: en este caso, sí que se nota diferencia con la original, quizá por el tipo de canción o el tipo de instrumentos que se están utilizando. No se pierde casi calidad, pero sí se nota diferencia respecto a \textsc{vorbis} y \textsc{opus}.
	\item \textsc{vorbis} y \textsc{opus}, 32, 16 y 8 bits: son iguales a la versión original, no se aprecia diferencia.
	\item \textsc{mp3}, 8 bits: se escucha bastante mal, la melodía se escucha más grave, como si hubiese sido distorsionada.
	\item \textsc{aac}, 8 bits: al igual que con 16 y 32 bits, se escucha muy similar a la versión original, no hay diferencias demasiado significativas aunque sí se aprecia un leve cambio en la tonalidad.
	\item \textsc{wma}, 8 bits: se escucha muy mal, hay mucho ruido y se escucha la voz muy grave y baja, como si hubiese disminuido el volumen de la voz y aumentado el del ruido de fondo.
\end{enumerate}

\subsection{Smells like teen spirit}

\begin{itemize}
	\item Canción: Smells like teen spirit
	\item Artista: Nirvana
	\item Género: Grunge
	\item Frecuencia de muestreo: 44.1 kHz
	\item Resolución: 16 bits
	\item Duración: 4:33
\end{itemize}

\begin{table}[H]
\centering
\begin{tabular}{|c|c|c|c|c|c|}
\hline
\diagbox[width=15em]{Resolución}{\\\textit{Códec}} & \textsc{ mp3 } & \textsc{ aac } & \textsc{ wma } & \textsc{vorbis} & \textsc{ opus } \\ \hline

32 bits & \makecell{2º\\ \includegraphics[width=0.03\textwidth]{b.png}} & \makecell{1º\\ \includegraphics[width=0.03\textwidth]{mb.png}} & \makecell{2º\\ \includegraphics[width=0.03\textwidth]{b.png}} & \makecell{1º\\ \includegraphics[width=0.03\textwidth]{mb.png}} & \makecell{1º\\ \includegraphics[width=0.03\textwidth]{mb.png}} \\ \hline
 
16 bits & \makecell{2º\\ \includegraphics[width=0.03\textwidth]{b.png}} & \makecell{1º\\ \includegraphics[width=0.03\textwidth]{mb.png}} & \makecell{2º\\ \includegraphics[width=0.03\textwidth]{b.png}} & \makecell{1º\\ \includegraphics[width=0.03\textwidth]{mb.png}} & \makecell{1º\\ \includegraphics[width=0.03\textwidth]{mb.png}} \\ \hline
 
 8 bits & \makecell{3º\\ \includegraphics[width=0.03\textwidth]{r.png}} & \makecell{2º\\ \includegraphics[width=0.03\textwidth]{b.png}} & \makecell{4º\\ \includegraphics[width=0.03\textwidth]{m.png}} & \makecell{1º\\ \includegraphics[width=0.03\textwidth]{mb.png}} & \makecell{1º\\ \includegraphics[width=0.03\textwidth]{mb.png}} \\ \hline
\end{tabular}
\caption{Tabla comparativa}
\label{my-label}
\end{table}

\subsubsection{Explicaciones a las valoraciones}
\begin{enumerate}
	\item \textsc{mp3} y \textsc{wma}, 32 y 16 bits: se escuchan similares a la original aunque con matices distintos. Son muy similares y tienen buena calidad aunque no tan buena como la original.
	\item \textsc{aac, vorbis} y \textsc{opus}, 16 y 32 bits: iguales a la original, no hay diferencia.
	\item \textsc{mp3}, 8 bits: se escucha regular, mejor que otras canciones para esta misma resolución pero se nota como ha perdido calidad de la original.
	\item \textsc{aac}, 8 bits: pierde un poco de calidad con respecto a la original o al mismo \textit{códec} con otras resoluciones, pero se sigue escuchando muy bien, la diferencia es poco llamativa.
	\item \textsc{wma}, 8 bits: de nuevo, al igual que en los casos anteriores, hay mucho ruido de fondo, pierde mucha calidad respecto a la original y respecto a otros \textit{códecs} para la misma resolución.
	\item \textsc{vorbis} y \textsc{opus}, 8 bits: iguales a la original, no se aprecian cambios.
\end{enumerate}

\subsection{Animales de laboratorio}

\begin{itemize}
	\item Canción: Animales de laboratorio
	\item Artista: Ska-P
	\item Género: Ska-Punk
	\item Frecuencia de muestreo: 44.1 kHz
	\item Resolución: 16 bits
	\item Duración: 4:48
\end{itemize}

\begin{table}[H]
\centering
\begin{tabular}{|c|c|c|c|c|c|}
\hline
\diagbox[width=15em]{Resolución}{\\\textit{Códec}} & \textsc{ mp3 } & \textsc{ aac } & \textsc{ wma } & \textsc{vorbis} & \textsc{ opus } \\ \hline

32 bits & \makecell{1º\\ \includegraphics[width=0.03\textwidth]{mb.png}} & \makecell{1º\\ \includegraphics[width=0.03\textwidth]{mb.png}} & \makecell{1º\\ \includegraphics[width=0.03\textwidth]{mb.png}} & \makecell{1º\\ \includegraphics[width=0.03\textwidth]{mb.png}} & \makecell{1º\\ \includegraphics[width=0.03\textwidth]{mb.png}} \\ \hline
 
16 bits & \makecell{1º\\ \includegraphics[width=0.03\textwidth]{mb.png}} & \makecell{1º\\ \includegraphics[width=0.03\textwidth]{mb.png}} & \makecell{1º\\ \includegraphics[width=0.03\textwidth]{mb.png}} & \makecell{1º\\ \includegraphics[width=0.03\textwidth]{mb.png}} & \makecell{1º\\ \includegraphics[width=0.03\textwidth]{mb.png}} \\ \hline
 
 8 bits & \makecell{3º\\ \includegraphics[width=0.03\textwidth]{r.png}} & \makecell{2º\\ \includegraphics[width=0.03\textwidth]{mb.png}} & \makecell{4º\\ \includegraphics[width=0.03\textwidth]{m.png}} & \makecell{1º\\ \includegraphics[width=0.03\textwidth]{mb.png}} & \makecell{1º\\ \includegraphics[width=0.03\textwidth]{mb.png}} \\ \hline
\end{tabular}
\caption{Tabla comparativa}
\label{my-label}
\end{table}

\subsubsection{Explicaciones a las valoraciones}
\begin{enumerate}
	\item \textsc{mp3, aac, wma, vorbis} y \textsc{opus}, 32 y 16 bits: en esta canción no se aprecian diferencias con la original, quizá porque el estilo de la canción de por sí es ciertamente ruidoso y las pequeñas variaciones que se hubiesen producido no son apreciables.
	\item \textsc{mp3}, 8 bits: en este caso sí se aprecia diferencia; la canción se hace más grave y se pierden timbres más agudos que sí estaban en la original.
	\item \textsc{aac}, 8 bits: practicamente igual a la original, se pierden algunos timbres pero casi imperceptible.
	\item \textsc{wma}, 8 bits: mucha diferencia con la original, hay ruido y se pierden más timbres que con \textsc{mp3}, definitivamente el peor.
	\item \textsc{vorbis} y \textsc{opus}, 8 bits: se escuchan exactamente igual a la original, no hay variación apreciable.
\end{enumerate}

\subsection{Conclusiones}

Como se ha podido comprobar tras este estudio, los \textit{códecs} en general, con una resolución de 16 o 32 bits (mejor de 16 puesto que ocupa menos espacio y se escuchan exactamente igual) las canciones no pierden practicamente nada o nada de la calidad original, por lo que merece la pena reducir el espacio que ocupan si conseguimos que la calidad sea la misma.\\

No obstante, utilizando resoluciones de 8 bits, la cosa cambia. \textit{Códecs} clásicos como \textsc{mp3} y \textsc{wma} pierden mucha calidad (en algunos casos maś que en otros, depende del tipo de sonidos que tenga la canción) con respecto a la original, mientras que los \textit{códecs} más recientes como \textsc{vorbis} u \textsc{opus} mantienen la misma calidad siempre y no se aprecian diferencias con respecto a la pista de sonido inicial.\\

A pesar de todo, esto es subjetivo y depende mucho de la capacidad auditiva de cada uno. Según lo que yo he ido escuchando en las distintas pistas y con todas las modificiaciones, mi clasificación de mejor a peor \textit{códec} sería la siguiente:

\begin{enumerate}
	\item \textsc{vorbis} y \textsc{opus}, al mismo nivel.
	\item \textsc{aac}, en segundo puesto.
	\item \textsc{mp3}, ya que mantiene mejor la calidad que \textsc{wma} según he comprobado.
	\item \textsc{wma}, sin duda, para mi gusto, el que pierde más calidad.
\end{enumerate}

\newpage

\section{Software utilizado}

Para poder cambiar los parámetros de digitalización y para poder cambiar los distintos \textit{códecs}, se ha utilizado \textbf{\textit{SoundConverter}}, que permite modificar y generar pistas de audio como se desee, con distintas frecuencias de muestreo (hay más frecuencias de las que se han utilizado en las tablas del apartado de parámetros de digitalización).\\

\begin{itemize}
	\item \textcolor{blue}{\url{https://soundconverter.org/}}
\end{itemize}

\section{Bibliografía}

\begin{itemize}
	\item Transparencias de teoría de Sistemas Multimedia: el sonido
	\item \footnotesize{\textcolor{blue}{\url{https://www.maketecheasier.com/convert-flac-to-mp3-easily-with-soundconverter/}}}
\end{itemize}

\end{document}